\documentclass[10pt,a4paper]{beamer}
\usepackage[utf8]{inputenc}
\usetheme{Boadilla}
\usepackage{amsmath}
\usepackage{amsfonts}
\usepackage{graphicx}
\usepackage{tikz}
\usetikzlibrary{snakes}
\usepackage{amssymb}
\usepackage{enumitem}
\title{Semaine 11 : Marché des titres adossés à des créances \\ $(1^{e}$ partie)}
\date{2021-03-31} 
\author{Simon-Pierre}
\institute{Université Laval}

\begin{document}
\begin{frame}
\titlepage
\end{frame}

\begin{frame}
\tableofcontents
\end{frame}

\section{Création des prêts hypothécaires résidentiels}
\begin{frame}{Initiateur de l'hypothèque}
\begin{itemize}[label=\bullet]
\item Le prêteur d'origine est appelé l'initiateur de l'hypothèque.
\vspace{0.5cm}
\item Les principaux initiateurs des prêts hypothécaires résidentiels sont les banques commerciales et les banquiers hypothécaires.
\vspace{0.5cm}
\item Les initiateurs de prêts hypothécaires peuvent assurer le service des prêts hypothécaires qu'ils ont émis, pour lesquels ils obtiennent des frais de gestion.
\end{itemize}
\end{frame}


\begin{frame}{Initiateur de l'hypothèque}
\begin{itemize}[label=\bullet]
\item Lorsqu'un initiateur d'hypothèque a l'intention de vendre l'hypothèque, il obtient un engagement de l'investisseur potentiel (acheteur).
\vspace{0.5cm}
\item Les entreprises publiques (Government-sponsored enterprise) et plusieurs sociétés privées achètent des prêts hypothécaires.
\vspace{0.5cm}
\item Lorsqu'une hypothèque est utilisée comme garantie par l'initiateur pour l'émission d'un titre financier, l'hypothèque est dite \textbf{titrisée}.
\end{itemize}
\end{frame}


\begin{frame}{Normes de souscription}
Les initiateurs peuvent générer des revenus pour eux-mêmes d'une ou de plusieurs manières.
\begin{enumerate}[label=\arabic*)]
\item Facturent des frais de création
\item Profit qui pourrait être généré par la vente d'un prêt hypothécaire à un prix plus élevé que son coût initial (secondary marketing profit)
\item L'initiateur de l'hypothèque peut détenir l'hypothèque dans son portefeuille de placements.
\end{enumerate}
\end{frame}

\begin{frame}{Normes de souscription}
Les initiateurs d'hypothèques peuvent soit
\begin{enumerate}[label=\arabic*)]
\item Détenir l'hypothèque dans leur portefeuille
\item Vendre l'hypothèque à un investisseur qui souhaite détenir l'hypothèque ou qui placera l'hypothèque dans un pool d'hypothèques à utiliser comme garantie pour l'émission d'un titre
\item Utiliser l'hypothèque comme garantie pour l'émission d'un titre.
\end{enumerate}
\end{frame}

\begin{frame}{Normes de souscription}
\begin{block}{Titrisation}
\begin{itemize}[label=\bullet]
\item Lorsqu'une hypothèque est utilisée comme garantie pour l'émission d'un titre, l'hypothèque est dite titrisée.
\item Une hypothèque conforme est celle qui répond aux normes de souscription établies par ces agences pour faire partie d'un pool de prêts hypothécaires sous-jacents à un titre qu'elles garantissent.
\item Si un demandeur ne satisfait pas aux normes de souscription, l'hypothèque est appelée hypothèque non conforme.
\end{itemize}
\end{block}
\end{frame}

\section{Ratios importants}

\begin{frame}{Ratio Payment-to-Income}
\begin{itemize}[label=\bullet]
\item Le ratio Payment-to-Income (PTI) est le rapport entre les paiements mensuels et revenu mensuel, qui mesure la capacité du demandeur à effectuer des paiements mensuels (à la fois des paiements hypothécaires et de l'impôt foncier).
\vspace{0.5cm}
\item Plus le PTI est bas, plus la probabilité que le demandeur soit en mesure de payer les mensualités requises est grande.
\end{itemize}
\end{frame}

\begin{frame}{Ratio Loan-to-Value}
\begin{itemize}[label=\bullet]
\item Le ratio Loan-to-Value (LTV) est le rapport entre le montant du prêt et la valeur marchande (ou estimée) de la propriété.
\vspace{0.5cm}
\item Plus ce ratio est bas, plus la protection du prêteur est élevée si le demandeur fait défaut et que le prêteur doit reprendre possession et vendre la propriété.
\vspace{0.5cm}
\item Le ratio Loan-to-Value a été trouvé dans de nombreuses études comme le déterminant le plus important de la probabilité de défaut de paiement.
\vspace{0.5cm}
\item Les données sur le comportement des emprunteurs indiquent qu'ils ont une propension accrue à arrêter volontairement leurs versements hypothécaires une fois que la LTV actuelle dépasse 125\%, même s'ils peuvent se permettre d'effectuer des versements mensuels. (Défaut stratégique)
\end{itemize}
\end{frame}


\section{Types de prêts hypothécaires résidentiels}

\begin{frame}{Types de prêts hypothécaires résidentiels}
Les prêts hypothécaires résidentiels peuvent être classés selon les attributs suivants:
\begin{itemize}[label=\bullet]
\item Statut de privilège
\item Classification de crédit
\item Type de taux d'intérêt
\item Type d'amortissement
\item Garanties de crédit
\item Soldes des prêts
\item Pénalités pour paiement anticipé.
\end{itemize}
\end{frame}

\begin{frame}{Types de prêts hypothécaires résidentiels}
\begin{block}{Statut de privilège}
Le statut de privilège d’un prêt hypothécaire indique l’ancienneté du prêt en cas de liquidation forcée du bien par défaut du débiteur.
\end{block}
\begin{block}{Classification de crédit}
\begin{itemize}[label=\bullet]
\item Prêt préférentiel : un prêt octroyé à un emprunteur ayant une qualité de crédit élevée 
\item Prêt subprime : un prêt émis lorsque l'emprunteur a une qualité de crédit inférieure ou lorsque le prêt n'est pas le premier ayant un privilège sur la propriété
\end{itemize}
\end{block}
\end{frame}

\begin{frame}{Types de prêts hypothécaires résidentiels}
\begin{block}{Type de taux d'intérêt}
\begin{itemize}[label=\bullet]
\item Le taux d'intérêt que l'emprunteur accepte de payer, appelé taux du billet, peut être fixe ou changer pendant la durée du prêt.
\item Pour une hypothèque à taux fixe (\textbf{fixed-rate mortgage (FRM)}), le taux d'intérêt est fixé à la signature du prêt et reste inchangé pendant toute la durée du prêt.
\item Pour un prêt hypothécaire à taux variable (\textbf{adjustable-rate mortgage (ARM)}),  le taux du billet change au cours de la durée du prêt.
\begin{itemize}[label=\bullet]
\item Le taux est basé à la fois sur le mouvement d'un taux sous-jacent et sur une marge prédéterminée.
\end{itemize}
\end{itemize}
\end{block}
\end{frame}

\begin{frame}{Types de prêts hypothécaires résidentiels}
\begin{block}{Type de taux d'intérêt}
\begin{itemize}[label=\bullet]
\item Le montant du paiement mensuel qui représente le remboursement du capital emprunté est appelé l'amortissement.
\item Les prêts hypothécaires à taux fixe et à taux variable amortissent entièrement les prêts.
\item Les prêts à taux fixe entièrement amortis (\textbf{fully amortizing loans}) ont un paiement qui est constant sur la durée du prêt. 
\end{itemize}
\end{block}
\begin{block}{Garanties de crédit}
Les prêts hypothécaires peuvent être classés selon l'entitée garantissant le prêt: 
\begin{itemize}[label=\bullet]
\item le gouvernement fédéral
\item une entreprise parrainée par le gouvernement 
\item une entité privée. 
\end{itemize}
\end{block}
\end{frame}



\section{Calcul du Paiement et solde hypothécaire}

\begin{frame}{Paiement hypothécaire mensuel}
La formule de calcul du paiement hypothécaire mensuel est

\begin{align*}
MP=MB_0 \left[ \frac{i(1+i)^n}{(1+i)^n-1} \right]
\end{align*}
Où
\begin{itemize}[label=\bullet]
\item $MP =$ versement hypothécaire mensuel (\$)
\item $MB_0 =$ solde hypothécaire initial (\$)
\item $i =$ taux hypothécaire divisé par 12 (en décimal)
\item $n =$ nombre de mois sur la durée du prêt hypothécaire.
\end{itemize}
\end{frame}

\begin{frame}{Solde hypothécaire restant}
Pour calculer le solde hypothécaire restant à la fin d'un mois, la formule suivante est utilisée:

\vspace{0.5cm}

\begin{align*}
MB_t=MB_0 \left[ \frac{(1+i)^n-(1+i)^t}{(1+i)^n-1} \right]
\end{align*}

\vspace{0.5cm}

Où $MB_t=$solde hypothécaire après t mois.

\end{frame}

\begin{frame}{Solde hypothécaire restant}
Pour calculer la partie du paiement hypothécaire mensuel qui correspond au paiement du capital prévu pour un mois, la formule suivante est utilisée: 

\vspace{0.5cm}

\begin{align*}
SP_t=MB_0 \left[ \frac{i(1+i)^{t-1}}{(1+i)^n-1} \right]
\end{align*}

\vspace{0.5cm}
Où $SP_t=$ remboursement du principal prévu pour le mois $t$.

\end{frame}

\section{Risques des prêts hypothécaire}


\begin{frame}{Risques des prêts hypothécaire}
Les investisseurs font face à quatre risques principaux en investissant dans des prêts hypothécaires résidentiels: 
\begin{enumerate}[label=\arabic*)]
\item Risque de crédit
\item Risque de liquidité
\item Risque de prix 
\item Risque de prix 
\end{enumerate}
\end{frame}

\begin{frame}{Risques des prêts hypothécaire}
\begin{block}{Le risque de crédit}
Le risque de crédit est le risque que le propriétaire / emprunteur fasse défaut. 
\end{block}
\begin{block}{Risque de liquidité}
Les prêts hypothécaires ont tendance à être plutôt illiquides parce qu'ils sont importants et indivisibles.
\end{block}
\begin{block}{Risque de prix}
\begin{itemize}[label=\bullet]
\item Le prix d'un instrument à revenu fixe évoluera dans une direction opposée aux taux d'intérêt du marché. 
\item Une hausse des taux d'intérêt fera baisser le prix d'un prêt hypothécaire.
\end{itemize}
\end{block}
\end{frame}


\begin{frame}{Risques des prêts hypothécaire}
Les trois composantes du flux de trésorerie sont: 
\begin{itemize}[label=\bullet]
\item Les intérêts
\item Remboursement du principal (remboursement ou amortissement prévu du principal)
\item Le prépaiement
\end{itemize}

\begin{block}{Paiements anticipés et incertitude des flux de trésorerie}
\begin{itemize}[label=\bullet]
\item Le risque de remboursement anticipé est le risque associé aux flux de trésorerie d’un prêt hypothécaire en raison des remboursements anticipés.
\item Les investisseurs craignent que les emprunteurs remboursent un prêt hypothécaire lorsque les taux hypothécaires en vigueur tombent en dessous du taux du prêt.
\end{itemize}
\end{block}
\end{frame}

\end{document}