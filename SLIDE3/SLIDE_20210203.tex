\documentclass[11pt]{beamer}
\usetheme{Boadilla}
\usepackage[utf8]{inputenc}
\usepackage{amsmath}
\usepackage{amsfonts}
\usepackage{amssymb}
\date{}
\author{Simon-Pierre Boucher \\ 3 Février 2021}
\title{Structure des taux d’intérêt}
%\setbeamercovered{transparent} 
%\setbeamertemplate{navigation symbols}{} 
%\logo{} 
%\institute{} 
%\date{} 
%\subject{} 
\begin{document}

\begin{frame}
\titlepage
\end{frame}

\begin{frame}{Relation risque-rendement}
\begin{itemize}
\item Un des principes les plus fondamental en finance est qu'un actif qui est perçu comme étant plus risqué,  devra forcément offrir un rendement plus élevé. 
\item Le taux de rendement d’une obligation devrait donc refléter son risque. 
\item Nous allons nous concentrer sur le risque d'échéance.
\begin{itemize}
\item Si le risque d’un flux monétaire change en fonction de son échéance,  alors le taux d’actualisation devrait aussi changer selon l’échéance.
\end{itemize}
\end{itemize}
\end{frame}
\begin{frame}{Composantes du taux de rendement}
Nous allons maintenant séparer le taux de rendement d'une obligation en deux composantes,  soit le taux de rendement de référence et la prime de risque.

\begin{enumerate}
\item Taux de rendement de référence (base interest rate): Taux que l’on pourrait obtenir sur un titre autrement identique mais sans risque.
\begin{itemize}
\item Marché domestique: Titres du gouvernement du Canada;
\item Marchés internationaux: Bons du Trésor américain;
\item Peut aussi être choisi pour isoler un risque particulier.
\end{itemize}
\item Prime de risque (risk premium) ou écart de taux (yield spread): Prime de rendement qui reflète le risque de l’obligation.
\end{enumerate}
\end{frame}
\begin{frame}{Mesures de la prime de risque}
On peut mesurer la prime de risque de trois façons suivante:
\begin{enumerate}
\item Prime de risque 
\begin{align*}
y^{obl}-y^{ref}
\end{align*}
\item Prime de risque relative
\begin{align*}
\frac{y^{obl}-y^{ref}}{y^{ref}}
\end{align*}
\item Ratio de taux
\begin{align*}
\frac{y^{obl}}{y^{ref}}
\end{align*}
\end{enumerate}
où 
\begin{itemize}
\item $y^{obl}=$ rendement de notre obligation
\item $y^{ref}=$ rendement sans risque
\end{itemize}
\end{frame}

\begin{frame}{Courbe des rendements à l’échéance}
\begin{itemize}
\item La courbe des rendements à l'échéance est une représentation graphique nous permettant de voir le risque d'une obligation en terme d'échéance.
\item Cela nous permet de voir la relation qui existe entre les taux de rendement de plusieurs obligations ayant un risque de crédit similaire et leur échéance.  
\item Historiquement,  la courbe des rendements à l’échéance est croissante plutôt que décroissante.
\begin{itemize}
\item Lorsque l'économie se porte bien,  la présence d'une courbe des rendements à l’échéance croissante est encore plus probable. 
\end{itemize}
\item Plus l’échéance est éloignée,  plus l’obligation est volatile et donc plus le taux de rendement exigé devrait être élevé.  
\end{itemize}
\end{frame}

\begin{frame}{Courbe des rendements à l’échéance}
\begin{itemize}
\item La courbe des rendements à l'échéance est souvent utilisée pour déterminer le taux de rendement permettant d’évaluer la valeur théorique d’une obligation non incluse dans la courbe. 
\item Il s'agit d'une méthode problématique étant donnée que des obligations ayant la même échéance peuvent avoir des taux de coupons différents.
\begin{itemize}
\item Sachant qu'une obligation ayant un taux de coupon plus élevé est moins volatile. 
\item Il s'agit donc d'une obligation avec un risque plus faible et par le fait même une obligaiton qui offre plus petit rendement.  
\item On peut donc avoir deux obligations de même échéance,  avec des rendements qui diffèrent.
\end{itemize}
\item Une  façon de rendre la courbe des rendements à l'échéance robuste,  est de construire une courbe basée sur des obligations ayant le même taux de coupon.
\end{itemize}
\end{frame}
\begin{frame}{Taux de rendement au comptant}
\begin{itemize}
\item Le taux de rendement au comptant $z_t$ est le taux d’une obligation à escompte pure d’échéance $t$ périodes. 
\item Le taux est appelé également spot rate.  
\item Il n’existe pas d’obligations à escompte pure pour toutes les échéances. 
\begin{itemize}
\item  Les taux de rendement au comptant sont construits à l’aide d’une théorie simplificatrice.\end{itemize} 
\end{itemize}
\end{frame}
\begin{frame}{Structure à terme des taux d’intérêt}
\begin{itemize}
\item La structure à terme des taux d'intérêts représente l'ensemble des taux de rendement au comptant pour différentes échéances ($z_t$). 
\item En utilisant cette stucture,  il nous sera possible de construire la courbe théorique des rendements au comptant à l’échéance. 
\item La représentation des $z_t$ dans la courbe devra utiliser un taux nominale annuel.
\end{itemize}
\end{frame}
\begin{frame}{Construction des $z_t$}
\begin{itemize}
\item On sait déja qu'une obligation réguilière est composé de flux monétaires $(FM)$. 
\item Si nous supposons que chaque flux monétare d'une obligations régulières sont prisent individuellement,  il est possible de dire qu'une obligation régulière est équivalent à plusieurs obligations zéro-coupon. 
\item Chaque flux monétaire représentera un obligation zéro-coupon et le flux monétaire en question sera le proxy de la veleur à échéance $M$.
\end{itemize}
\end{frame}

\begin{frame}{Construction des $z_t$}
Selon la théorie de l’absence d’arbitrage, la valeur de l’obligation à coupons devrait être égale à la somme de la valeur des obligations à escompte pure:
\begin{align*}
P=\frac{C}{(1+z_1)^1}+\frac{C}{(1+z_2)^2}+\frac{C}{(1+z_3)^3}+...+\frac{C}{(1+z_n)^n}+\frac{M}{(1+z_n)^{n}}
\end{align*}
\begin{itemize}
\item En supposant qu’il y a absence d’arbitrage,  il est possible de résoudre l’équation précédente afin de trouver les valeurs de $z_1, z_2, z_3,...,z_n$.  
\item Pour trouver les taux au comptant pour toutes les échéances,  il faut procéder à l’aide d’une méthode récursive (bootstrapping method). 
\item S’il manque des échéances, il faut utiliser une technique de lissage, telle que l’interpolation linéaire, pour déterminer les taux manquants.
\end{itemize}
\end{frame}
\begin{frame}{Courbe des rendements à l’échéance au pair}
\begin{itemize}
\item Courbe des rendements à l’échéance utilisant des obligations de référence ayant été transformées pour être évaluées au pair.  
\item Cette courbe est utilisée en pratique à la place de la courbe des rendements à l’échéance pour réduire l’effet du taux de coupon.
\item Pour chaque échéance $n$,  cette courbe utilise le taux de coupon correspondant au coupon $C$ déterminé à l’aide de l’équation suivante:
\end{itemize}
\begin{align*}
M=\sum_{t=1}^n \left[ \frac{C}{(1+z_t)^t} \right]+\frac{M}{(1+z_n)^{n}}
\end{align*}
\begin{itemize}
\item On peut donc voir que pour la courbe des rendements à l’échéance au pair on replace dans la formule le prix de l'obligation $P$ par la valeur à l'échéance $M$.
\end{itemize}
\end{frame}
\begin{frame}{Taux de rendement à terme}
\begin{itemize}
\item $f_{t,n}$ représente le taux de rendement à terme pour une échéance de $n$ périodes à compter de la période $t$.  
\item Le taux de rendement à terme est le taux de rendement (effectif par période) implicite entre les périodes $t$ et $t+n$ étant donné les taux de rendement au comptant pour les échéances t et t+n périodes,  $z_t$ et $z_{t+n}$.
\item Les taux de rendement à terme représentent les taux de rendement au comptant futurs extraits des taux de rendement au comptant actuels.
\end{itemize}
\end{frame}
\begin{frame}{Trouver $f_{t,n}$}
Le taux de rendement à terme $t_{f,n}$ est le taux qui rend un investisseur indifférent entre les deux situations suivantes:
\begin{itemize}
\item Investir de $0$ à $t+n$ à $z_{t+n}$
\item Investir de $0$ à $t$ à $z_t$ et de $t$ à $t+n$ à $f_{t,n}$
\end{itemize}
Grâce à cette dernière hypothèse, on peut poser la formule suivante
\begin{align*}
(1+z_{t+n})^{t+n}=(1+z_t)^t \times (1+f_{t,n})^n
\end{align*}
\begin{itemize}
\item L’investisseur se sert de son anticipation du taux au comptant d’échéance $n$ périodes au temps $t$,  $E(z_n)$,  par rapport au taux à terme $f_{t,n}$,  pour décider s’il doit investir jusqu’à $t$ au taux $z_t$ ou jusqu’à $t+n$ au taux $z_{t+n}$.  
\item  Il peut,  entre autre,  se garantir un taux $f_{t,n}$ en empruntant à $z_t$ pour investir à $z_{t+n}$.
\end{itemize}
\end{frame}

\begin{frame}{Exemple 1}
Les deux investissements suivants sont équivalents
\begin{itemize}
\item Investir pendant 4 périodes à un taux au comptant $z_4$
\item La combinaison suivante
\begin{itemize}
\item Investir pendant 1 période à un taux au comptant $z_1$
\item Investir pendant 1 période dans 1 période au taux à terme $f_{1,1}$
\item Investir pendant 1 période dans 2 périodes au taux à terme $f_{2,1}$
\item Investir pendant 1 période dans 3 périodes au taux à terme $f_{3,1}$
\end{itemize}
\end{itemize}
\begin{align*}
(1+z_4)^4=(1+z_1) \times (1+f_{1,1}) \times (1+f_{2,1}) \times (1+f_{3,1})
\end{align*}
\end{frame}

\begin{frame}{Exemple 2}
Les deux investissements suivants sont équivalents
\begin{itemize}
\item Investir pendant 4 périodes à un taux au comptant $z_4$
\item La combinaison suivante
\begin{itemize}
\item Investir pendant 1 période à un taux au comptant $z_1$
\item Investir pendant 2 périodes dans 1 période au taux à terme $f_{2,1}$
\item Investir pendant 1 période dans 3 périodes au taux à terme $f_{3,1}$
\end{itemize}
\end{itemize}
\begin{align*}
(1+z_4)^4=(1+z_1) \times (1+f_{1,2})^2 \times (1+f_{3,1})
\end{align*}
\end{frame}

\begin{frame}{Exemple 3}
Les deux investissements suivants sont équivalents
\begin{itemize}
\item Investir pendant 4 périodes à un taux au comptant $z_4$
\item La combinaison suivante
\begin{itemize}
\item Investir pendant 1 période à un taux au comptant $z_1$
\item Investir pendant 1 période dans 1 période au taux à terme $f_{1,1}$
\item Investir pendant 2 périodes dans 2 périodes au taux à terme $f_{2,2}$
\end{itemize}
\end{itemize}
\begin{align*}
(1+z_4)^4=(1+z_1) \times (1+f_{1,1}) \times (1+f_{2,2})^2
\end{align*}
\end{frame}

\begin{frame}{Exemple 4}
Les deux investissements suivants sont équivalents
\begin{itemize}
\item Investir pendant 4 périodes à un taux au comptant $z_4$
\item La combinaison suivante
\begin{itemize}
\item Investir pendant 2 périodes à un taux au comptant $z_2$
\item Investir pendant 1 période dans 2 périodes au taux à terme $f_{2,1}$
\item Investir pendant 1 période dans 3 périodes au taux à terme $f_{3,1}$
\end{itemize}
\end{itemize}
\begin{align*}
(1+z_4)^4=(1+z_2)^2 \times (1+f_{2,1}) \times (1+f_{3,1})
\end{align*}
\end{frame}


\begin{frame}{Formes de la structure à terme des taux}
\begin{itemize}
\item Plusieurs théories ont été mit de l'avant pour essayer d'expliquer la forme de la structure à terme des taux à travers le temps. 
\item Historiquement,  les trois formes suivantes ont été observées.
\begin{itemize}
\item Croissante ou normale
\item Décroissante ou inversée
\item Horizontale ou plate
\end{itemize}
\end{itemize}
\end{frame}

\begin{frame}{Théorie des anticipations pures}
\begin{itemize}
\item La théorie des anticipations pures montre que les taux de rendement à terme sont des estimateurs non biaisés des taux au comptant leur étant associés dans le futur. 
\item De façon simple, on peut imaginer un monde avec 2 périodes. La première période allant de $t=0$ à $t=1$ et la deuxième période allant de $t=1$ à $t=2$.
\item Si on pose l'hypothèse que la structure à terme est croissante,  alors le rendement que je vais avoir en achetant une obligation en $t=0$ et venant à échéance en $t=1$ sera plus faible que le rendement obtenu avec une obligation achetée en $t=0$ et venant à échéance en $t=2$.  
\end{itemize}
\end{frame}


\begin{frame}{Théorie des anticipations pures}
\begin{itemize}
\item Le taux d'une obligation achetée en $t=0$ et venant à échéance en $t=1$ est représenté par $z_{1}$,  alors que le taux d'une obligation achetée en $t=0$ et venant à échéance en $t=2$ est représenté par $z_{2}$. 
\item De plus, nous allons représenter le taux future d'une obligation qui serait achetée en $t=1$ et qui viendrait à échéance en $t=2$ par $E_0(z_{1,1})$.  
\item On peut dire que $E_0(z_{1,1})$ représente l'anticipation en $t=0$ du taux qui sera en vigueur dans une période et ce pour une période. 
\end{itemize}
\end{frame}



\begin{frame}{Théorie des anticipations pures}
Afin d'éviter la présence d'arbitrage, il doit y avoir un taux de rendement à terme $f_{1,1}$ qui rend l'équation suivante vrai.
\begin{align*}
(1+z_2)^2=(1+z_1) \times (1+f_{1,1})
\end{align*}
On peut donc exprimer $f_{1,1}$ en fonction de $z_1$ et $z_2$ de la façon suivante.
\begin{align*}
f_{1,1}=\frac{(1+z_2)^2}{(1+z_1)}-1
\end{align*}
\end{frame}

\begin{frame}{Théorie des anticipations pures}
En appliquant la définition de la théorie des anticipations pures à notre monde composé de 2 périodes, on arrive à l'énoncé: \\
\vspace{0.5cm}
 \textbf{La théorie des anticipations pures montre que le taux de rendement à terme $f_{1,1}$ est un estimateur non biaisés du taux au comptant $z_{1,1}$ lui étant associés dans le futur. }\\
 \vspace{0.5cm}
Ce dernier énoncé peut être représenté par l'équation suivante:
\begin{align*}
E_0(z_{1,1})=f_{1,1}=\frac{(1+z_2)^2}{(1+z_1)}-1
\end{align*}
\end{frame}

\begin{frame}{Théorie des anticipations pures}
\begin{itemize}
\item Dans la théorie des anticipations pures, les agents sont neutres face au risque et la structure à terme des taux d’intérêt reflète simplement les anticipations du marché quant aux taux au comptant à venir. 
\item Voici maintenant,  les propriétés de cette théorie.
\begin{itemize}
\item  Forme croissante $\rightarrow$ hausse future de $z_t$
\item Forme décroissante $\rightarrow$ baisse future de $z_t$
\item Forme horizontale $\rightarrow$ stabilité future de $z_t$
\end{itemize}
\end{itemize}
\end{frame}

\begin{frame}{Théorie de la prime de liquidité}
\begin{itemize}
\item La théorie de la prime de liquidité nous dit que le taux de rendement à terme est égal au taux de rendement au comptant anticipé pour la période correspondante auquel on ajoute une prime de liquidité qui augmente avec l’échéance.  
\item Si nous reprenons l'example de la théorie précédente.
\end{itemize}
\begin{align*}
(1+z_2)^2>(1+z_1) \times (1+f_{1,1})
\end{align*}
\begin{itemize}
\item Vous aurez remarqué qu'à la place d'une égalité, nous avons une inégalité.  
\end{itemize}
\end{frame}

\begin{frame}{Théorie de la prime de liquidité}
\begin{block}{Théorie des anticipations pures}
\begin{itemize}
\item Selon la théorie des anticipations pures  il y aura possibilité d'arbitrage en achetant un portefeuille payant un taux $z_2$ chaque années pendant deux ans et vendant un portefeuille payant un taux $z_1$ la première année et un taux $f_{1,1}$ la deuxième année. 
\end{itemize}
\end{block}
\begin{block}{Théorie de la prime de liquidité}
\begin{itemize}
\item La théorie de la prime de liquidité dirait qu'il ne s'agit pas nécessairement d'une opportunité d'arbitrage étant donnée que le portefeuille payant un taux $z_2$ représente un plus grand risque de liquidité. 
\end{itemize}
\end{block}
\end{frame}

\begin{frame}{Théorie de la prime de liquidité}
\begin{itemize}
\item Sachant que le risque doit être rénuméré,  il est normal qu'il offre un rendement supérieur.  
\item dans cette théorie,  les agents sont averses au risque et préfèrent les placements à court terme (plus liquides et moins volatils) aux placements à long terme. 
\item La structure à terme des taux d’intérêt reflète les anticipations du marché quant aux taux au comptant à venir ainsi que la prime de liquidité.
\end{itemize}
\end{frame}

\begin{frame}{Théorie de la segmentation du marché}
\begin{itemize}
\item Dans la théorie de la segmentation du marché la forme de la courbe est déterminée par l’offre et la demande des titres financiers pour chaque échéance.  
\item Les agents ont des horizons bien définis et n’ont aucune préférence pour les taux d’échéances autres que leurs horizons. 
\item La détermination des taux à court terme est indépendante de celle des taux à long terme puisque le marché à court terme est segmenté du marché à long terme.
\end{itemize}
\end{frame}
\begin{frame}{Théorie des habitats préférés}
\begin{itemize}
\item La théorie des habitats préférés est une combinaison de la théorie de la prime de liquidité et de la théorie de la segmentation du marché. 
\item Cette théorie suppose qu'il existe des primes de liquidité.
\item Puisque le marché est partiellement segmenté,  celles-ci ne sont pas nécessairement une fonction croissante de l’échéance.
\item Les agents sont prêts à sortir de leur habitat préféré si on leur offre une prime de liquidité suffisante.
\end{itemize}
\end{frame}
\end{document}